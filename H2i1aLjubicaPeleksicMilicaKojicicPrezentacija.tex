\documentclass[hyperref={pdfpagelabels=false}]{beamer}
\setbeamerfont{structure}{family=\rmfamily,shape=\itshape} 
\usecolortheme[named=purple]{structure}
\usetheme{Malmoe}
\setbeamertemplate{items}[ball] 
\setbeamertemplate{blocks}[rounded][shadow=true] 
\usepackage{lmodern}

\title{\textbf Asembler x86\_64 -- instrukcije op\v ste namene\\ \small{--- seminarski rad ---  \\Arhitektura i operativni sistemi}}   
\author{Ljubica Peleksi\v c i Milica Koji\v ci\' c} 
\date{\today} 

\begin{document}
%POCETAK PREZENTACIJE I STVARANJE SLAJDA SA NASLOVOM I SADRZAJEM
	\begin{frame}
		\titlepage
	\end{frame} 

 
 %UVODNI DEO PREZENTACIJE
\section{Uvod} 
%ASEMBLERSKI JEZIK
\subsection{Asemblerski jezik}

	\begin{frame}
		\frametitle{Asemblerski jezik} 
		\begin{itemize}
			\pause
			\item Svaki ra\v cunar ima svoju arhitekturu skupa instrukcija (Instruction Sat Architecture,
			ISA)koja se naziva \textbf{ma\v sinski jezik}\pause 
			\item Asemblerski jezik je simboli\v cka predstava te arhitekture\pause
			\item Binarni brojevi iz ma\v sinskog jezika su sada zamenjeni mnemoni\v ckim oznakama\pause
			\item Osim kratkih re\v ci(ADD, SUB, MUL) koje se koriste za ma\v sinske instrukcije, asembleri dozvoljavaju
			kori\v s\' cenje simboli\v ckih imena za konstante i oznaka koje se odnose na
			instrukcije i memorijske adrese
			\end{itemize}
	\end{frame}
	
	\begin{frame}[fragile]
		\frametitle{Za\v sto asembler?} 
			\pause
			\verb|"Asembler je program programa, alat svih alata i on je ozbiljno oruzje u rukama pravog programera"|\pause
		\begin{itemize} 
			\item Najbli\v za forma komunikacije izmedju \v coveka i ma\v sine\pause
			\item Prepravljanje koda za koji nemamo izvorni kod\pause
			\item Pisanje koda kriti\v cnih performansi\pause
			\item Pristupanje novim karakteristikama,pre nego \v sto ih uvedu u kompajler
		\end{itemize}
	\end{frame}

%ARHITEKTURA
\subsection{Arhitektura x86-64}
	\begin{frame}
		\frametitle{Arhitektura x86-64} 
		\pause
		\begin{itemize} 
			\item 64-bitni ra\v cunari zamenjuju 32-bitne \pause
			\item x64-ime za Intelovu i AMD-ovu nadogradnju 32-bitne arhitekture skupa instrukcija \pause
			\item AMD-ova prva verzija x64 se zvala x86-64, a zatim su je preimenovali u AMD64\pause
			\item Intel je svoju implementaciju nazvao IA-32e, a zatim EMT64\pause
			\item Ova arhitektura:\pause
			\begin{itemize}
				\item ima ve\' ce koli\v cine virtuelne i fizi\v cke memorije\pause
				\item ima 64-bitne registre\pause
				\item potpuno je unazad kompatibilna sa 16-bitnom i 32-bitnom arhitekturom\pause
				\item zadr\v zava set instrucija skoro u potpunosti\pause
				\item ima sli\v cne na\v cine adresiranja
			\end{itemize}
		\end{itemize}
	\end{frame}

	
%REGISTRI&ADRESIRANJA
\subsection{Organizacija registara i na\v cini adresiranja}
	%ORGANIZACIJA REGISTARA
	\begin{frame}[fragile]
		\frametitle{Organizacija registara}
		\begin{itemize}
			\item \pause Registri se u odnosu na one iz 32-bitnog asemblera pro\v siruju sa 32 na 64 bita\pause
			\item U 64-bitnoj arhitekturi se i dalje koriste 32-bitni,16-bitni i 8-bitni registri\pause
			\item
				\verb|U x86 asembleru postoji 8 primarnih registara: RAX(akumulator), RCX(brojac),RDX(registar podataka),RBX(bazni registar),RSP(pokazivac na stek),|
				\verb|RBP(pokazivac na narednu instrukciju) RSI i RDI.|\pause
			\item Floating-point registri\pause
			\item Registar RFLAGS je pro\v sirenje EFLAGS i kao i on,skaldi\v sti flagove,a sadr\v zaj se menja u zavisnosti od rezultata aritmeti\v ckih operacija.\pause
		\end{itemize}
	\end{frame}
	
	%NACINI ADRESIRANJA
	\begin{frame}
		\frametitle{Na\v cini adresiranja}
		\begin{itemize}
		\item \pause Na\v cini adresiranja su zapravo na\v cini na koje instrukcija mo\v ze da pristupi registrima ili memoriji\pause
		\item Na\v cini adresiranja:
			\begin{itemize}
				\item Neposredno adresiranje-operand u instrukciji je konstantan bajt ili re\v c \pause
				\item Direktno adresiranje-operand u instrukciji je adresa podatka \pause
				\item Indirektno registarsko-adresa operanda nalazi u nekom od registara RBX, RSI ili RDI.\pause
				\item Registarsko aresiranje sa pomerajem-konstanta se postavi ispred registra, adresa se pronalazi dodavanjem sadr\v zaja registra konstanti.\pause
				\item Registarsko indeksno podrazumeva da je u RDI po\v cetna adresa niza, a broja\v c u  RCX.\pause
				\item Registarsko indeksno sa pomerajem-kombinacija prethodna dva
			\end{itemize}
			\pause
		\end{itemize}
	\end{frame}
	
	
	
%DEO SA INSTRUKCIJAMA
\section{Instrukcije op\v ste namene}
	%ARITMERICKE INSTRUKCIJE
	\begin{frame}
		\frametitle{Aritmeti\v cke instrukcije} 
		\begin{itemize}
			\item \pause Instrukcija ADD vr\v si sabiranje i ima dva operanda: \pause \\
					ADD RAX, 4 \pause
			\item Instrukcija SUB vr\v si oduzimanje i ima dva operanda: \pause \\
					SUB RBX, x \pause
			\item Instrukcija ADC vr\v si sabiranje s prenosom i ima dva operanda:\pause \\
					ADC RAX, 4 \pause
			\item Instrukcija SBB vr\v si oduzimanje s pozajmicom i ima dva operanda:\pause \\
					SBB RBX, x \pause
			\item Instrukcija NEG vr\v si negaciju i ima jedan operand kome menja znak. \pause \\
					NEG X 
		\end{itemize}
	\end{frame}
	
	\begin{frame}
		\frametitle{Aritmeti\v cke instrukcije}
		\begin{itemize}
			\item \pause Instrukcije INC i DEC vr\v se (redom) uvecavanje i smanjivanje argumenta za jedan i imaju po jedan operand. \pause \\
							INC RDX (uve\' cava vrednost registra RDX za 1) \pause \\
							DEC RDX (smanjuje vrednost registra RDX za 1) \pause
			\item Razlikujemo dve instrukcije za mno\v zenje: MUL za neozna\v cene i IMUL za ozna\v cene brojeve. Instrukcija MUL ima samo jedan operand a 
					IMUL mo\v ze imati jedan operand i u tom slu\v caju je analogna instrukciji MUL, ali mo\v ze imati i dva i tri operanda.\pause \\
					MUL RBX \pause 
			\item Rezlikujemo dve instrukcije za deljenje: DIV se odnosi na deljenje neozna\v cenih, a IDIV na deljenje ozna\v cenih brojeva; Instrukcija
					DIV ima jedan operand.IDIV je analogna instrukciji DIV.\pause \\
					DIV RCX 
		\end{itemize}
	\end{frame}
	
	%LOGICKE INSTRUKCIJE
	\begin{frame}[fragile]
		\frametitle{Logi\v cke instrukcije}
		\begin{itemize}
			\item \pause Instrukcija AND vr\v si bitovsku konjukciju i ima dva operanda:\pause
			 \\ AND RDX, RCX\pause
			\item Instrukcija OR vr\v si bitovsku disjunkciju i ima dva operanda: \pause
			 \\ OR RDX, RCX \pause
			\item Instrukcija XOR vr\v si bitovsku ekskluzivnu disjunkciju i ima dva operanda:\pause
			 \\ XOR RDX, RCX \pause
			\item Instrukcija NOT vr\v si bitovksku negaciju i ima jedan operand: \pause
			 \\ NOT RDX
		\end{itemize}
	\end{frame}
	
	%INSTRUKCIJE ZA SIFTOVANJE I ROTACIJU
	\begin{frame}
		\frametitle{Instrukcije za \v siftovanje i rotaciju}
		\begin{itemize}
			\item \pause Logi\v cko \v siftovanje \pause
				\begin{itemize}
					\item Instrukcija SHR vr\v si \v siftovanje udesno i ima jedan operand\pause
					\item Instrukcija SHL vr\v si \v siftovanje ulevo i ima jedan operand\pause
				\end{itemize}
			\item Artimeti\v cko \v siftovanje
				\begin{itemize}
					\item Instrukcija SHR vr\v si \v siftovanje udesno i ima jedan argument \pause
					\item Instrukcija SHL vr\v si \v siftovanje ulevo i ima jedan argument \pause
				\end{itemize}
			\item Instrukcije za \v siftovanje sa prenosom
			\begin{itemize}
					\item Instrukcija SCR vr\v si \v siftovanje udesno sa prenososm i ima jedan argument\pause
					\item Instrukcija SCL vr\v si \v siftovanje ulevo sa prenososm i ima jedan argument\pause
				\end{itemize}
			\item Instrukcije za rotaciju
			\begin{itemize}
					\item Instrukcija ROR vr\v si rotaciju udesno i ima jedan operand\pause
					\item Instrukcija ROL vr\v si rotaciju ulevo i ima jedan operand 
				\end{itemize}
			
		\end{itemize}
	\end{frame}

	%INSTRUKCIJE ZA TRANSFER PODATAKA
	\begin{frame}
		\frametitle{Instrukcije za transfer podataka} 
		\begin{itemize}
			\item \pause Instrukcija MOV kopira vrednost drugog operanda u prvi pri \v cemu se nijedan fleg ne menja \pause
			\item Instrukcija XCHG zamenjuje vrednosti prvog i drugog operanda, pri \v cemu se takodje nijedan fleg ne menja\pause
			\item Instrukcije pri kopiranju drugog operanda u prvi nepopunjena mesta u prvom argumentu popunjavaju nulama ili znakom drugog argumenta:\pause
			\item Instrukcija MOVSB nema nijedan operand i kopira jedan bajt sa lokacije koja se nalazi u RSI registru na lokaciju koja se nalazi u RDI.
			Instrukcija takodje ne menja flegove
			\item MOVESW je instrukcija koja nema nijedan operand i kopira dva bajta (jednu re\v c) iz RSI u RDI
		\end{itemize}
	\end{frame}

	%INSTRUKCIJE ZA KONTROLU TOKA
	
	%PRVI SLAJD 
	\begin{frame}
		\frametitle{Instrukcije za kontrolu toka}
		\begin{itemize}	
			\item \pause Instrukcije za poredjenje \pause
				\begin{itemize}
					\item TEST je instrukcija koja izvr\v sava bitovski konjukciju nad dva operanda i postavlja flegove, ali ne \v cuva rezultat \pause
					\item CMP je instrukcija koja vr\v si oduzimanje izmedju dva operanda,postavlja flegove, ali ne \v cuva rezultat \pause
				\end{itemize}
			
			\item Instrukcije za poziv procedura\pause
				\begin{itemize}
					\item Instrukcija CALL bezuslovno ska\v ce na adresu koja je zadata kao operand, ali prethodno stavlja na vrh steka adresu povratka iz RIP
					registra \pause
					\item Instrukcija RET zavr\v sava potrogram, ona skida sa steka povratnu adresu, stavlja je u RIP registar i vr\v si bezuslovan skok na tu adresu.
				\end{itemize}
		\end{itemize}
	\end{frame}
			
	%DRUGI SLAJD	
	\begin{frame}
		\frametitle{Instrukcije za kontrolu toka}
		Skokovi
		\begin{itemize}
			\item Bezuslovni skok - JMP 
			\item Skok ako je jednako - JE
			\item Skok ako nije jednako - JNE
			\item Skok ako je ve\' ce - JG,JA i JGE,JAE
			\item Skok ako je manje - JL,JB i JLE,JBE
			\item Skok ako postoji prekora\v cenje - JO
			\item Skok u slu\v caju nule - JNZ i JZ		
		\end{itemize}
	\end{frame}
	
	%TRECI SLAJD
	\begin{frame}
		\frametitle{Instrukcije za kontrolu toka}
		\begin{itemize}
			\item Instrukcije broja\v cke petlje\pause
				\begin{itemize}
					\item Instrukcija LOOP ima samo jedan argument koji predstavlja adresu memorijske lokacije. Ona umanjuje vrednost registra RCX za jedan i ska\v ce
					na zadatu adresu sve dok vrednost registra RAX ne bude nula.LOOP ne postavlja flegove \pause
					\item Instrukcija LOOPX umanjuje vrednost registra RCX i ska\v ce na adresu zadatu operandom ako je njen uslov zadovoljen, tj. ako je postavljen
					odgovaraju\v ci fleg:
						\begin{itemize}
							\item LOOPE ska\v ce ako je jednako
							\item LOOPZ ska\v ce ako je nula
							\item LOOPNE ska\v ce ako nije jednako
							\item LOOPNZ ska\v ce ako nije nula
						\end{itemize}
				\end{itemize}
		\end{itemize}
	\end{frame}
				
	%CETVRTI SLAJD			
	\begin{frame}
		\frametitle{Instrukicje za kontrolu toka}
		\begin{itemize}
			\item Instrukcije ENTER i LEAVE\pause
				\begin{itemize}
					\item Instrukcija ENTER pravi stek okvir sa odredjenom koli\v cinom prostora alociranom na steku. Obi\v cno se taj prostor rezervi\v se za 
					lokalne promenljive koje se koriste u procedurama \pause
					\item Instrukcija LEAVE uni\v stava trenutni stek okvir i restaurira prethodni okvir \pause
				\end{itemize}
				
			\item Druge instrukcije kontrole\pause
				\begin{itemize}
					\item HLT instrukcija zaustavlja procesor \pause
					\item NOP zna\v ci da nema operacije. Ova instrukcija ne radi nista, tj. samo tro\v si instrukcioni ciklus \pause
					\item LOCK instrukcija isti\v ce procesorov signal LOCK koji je istaknut tokom operacije koja ide uz instrukciju\pause
					\item WAIT instrukcija ˇceka da procesor zavr\v si svoju poslednju operaciju
				\end{itemize}
		\end{itemize}
	\end{frame}
	
	%INSTRUKICJE ZA RAD SA STEKOM
	\begin{frame}
		\frametitle{Instrukcije za rad sa stekom}
		\begin{itemize}
			\item \pause Instrukcija PUSH smanjuje za 1 pokaziva\v c na stek i podatak koji se nalazi u argumentu stavlja na lokaciju na koju pokazuje 
			pokaziva\v c na stek:\\ \pause
			PUSH EAX \pause
			\item Instukcija POP stavlja podatke koji se nalaze na lokaciji na koju pokazuje pokaziva\v c na stek u argument,a zatim povecava pokaziva\v c
			na stek za 1:\pause \\
			POP EAX
			\item Instrukcija PUSHF smanjuje pokaziva\v c na stek za 1 a zatim na lokaciju na koju pokazuje pokaziva\v c na stek,stavlja sadr\v zaj flag 
			registra\pause
			\item Instrukcija POPF u flag regitar stavlja sadr\v zaj memorijske adrese na koju pokazuje pokaziva\v c na stek,a zatim pokaziva\v c na stek 
			uve\' cava za 1\pause
		\end{itemize}
	\end{frame}
	
	%INSTRUKCIJE ZA RAD SA FLAGOVIMA
	\begin{frame}
		\frametitle{Instrukcije za rad sa flagovima} 
		\begin{itemize}
			\item Fleg za prekide\pause 
				\begin{itemize}
					\item Instrukcija STI postavlja fleg za prekide; u tom slu\v caju procesor prihvata zahtev za prekid\pause
					\item Instrukcija CLI poni\v stava fleg za prekide\pause
				\end{itemize}
			\item Fleg za prenos
				\begin{itemize}
					\item Instrukcija STC postavlja fleg za prenos\pause
					\item Instrukcija CLC poni\v stava fleg za prenos\pause
					\item Instrukcija CMC komplementira fleg za prenos\pause
				\end{itemize}
			\item Ostali flag-ovi
				\begin{itemize}
					\item Instrukcija SAHF \v cuva sadr\v zaj AH registra u ni\v zim bajtovima fleg registra\pause
					\item Instrukcija LAHF stavlja u AH registar ono \v sto se nalazi u ni\v zim bajtovima fleg registra\pause
				\end{itemize}
		\end{itemize}
	\end{frame}
	
	
	%ULAZNO-IZLAZNE INSTRUKCIJE & SISTEMSKE INSTRUKCIJE
	\begin{frame}
		Ulazno-izlazne instrukcije
		\begin{itemize}
			\item \pause Instrukcija IN skoro uvek radi sa operandima AX i DX (tj EAX,EDX ili RAX,RDX).DX \v cesto drzi adresu porta za \v citanje,a AX prima
			podatke od porta\pause
			\item Instrukcija OUT je veoma sli\v cna instrukciji IN.OUT stavlja podatke iz registra koje je u prvom operandu,na port dat u drugom operandu
		\end{itemize}
		
		Sistemske instrukcije
		\begin{itemize}
			\item \pause Instrukcija SYSENTER uvodi procesor u za\v sti\' ceni re\v zim \pause
			\item Instrukcija SYSEXIT izvla\v ci procesor iz za\v sti\' cenog re\v zima i uvodi ga u re\v zim za korisnika
		\end{itemize}		
	\end{frame}

	

	
	
	
	
\end{document}