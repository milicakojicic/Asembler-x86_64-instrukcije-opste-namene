\documentclass[a4paper]{article}

\usepackage{color}
\usepackage{url}
\usepackage{graphicx}
\usepackage[utf8]{inputenc} 
\usepackage[unicode]{hyperref}
\hypersetup{colorlinks,citecolor=green,filecolor=green,linkcolor=blue,urlcolor=blue}
\usepackage[english,serbian]{babel}

\title{Asembler x86\_64 -- instrukcije op\v ste namene\\ \small{--- seminarski rad ---  \\Arhitektura i operativni sistemi}}
\author{Ljubica Peleksi\'c i Milica Koji\v ci\' c}
%\date{}
\date{5.~novembar 2013.}

\begin{document}

\maketitle
\abstract{Da bi bilo koji procesor mogao da se programira na asemblerskom jeziku, programer mora detaljno poznavati 
njegovu arhitekturu skupa instrukcija. Shodno tome, ovde \' cemo se baviti arhitekturom x86\_64 i to: organizacijom
memorije, na\v cinima adresiranja, a najvise skupom instrukcija. Instrukcije obi\v cno imaju jedan,
dva ili tri operanda, kojima se mo\v ze pristupiti neposrednim, direktnim, registarskim, indeksnim 
ili nekim drugim na\v cinom adresiranja. Neki ra\v cunari imaju veliki broj slo\v zenih na\v cina adresiranja.
U op\v stem slu\v caju postoje instrukcije za preme\v stanje podataka, za binarne i unarne operacije, 
uklju\v cuju\' ci i aritmeti\v cke i logi\v cke operacije, instrukcije za grananje, pozivanje procedura,
rad u petlji, kao i ulazno-izlazne instrukcije. Tipi\v cne instrukcije preme\v staju re\v c iz memorije 
u registar (ili obrnuto), sabiraju, oduzimaju, mno\v ze ili dele sadr\v zaj dva registra ili sadr\v zaj 
registra i memorijske re\v ci, ili porede sadr\v zaj registara ili memorije. U narednim odeljcima razmotri\' cemo one koje 
najvi\v se koristimo pri programiranju u asembleru x86\_64.}
\tableofcontents

\newpage


\section{\textbf{UVOD}} 

\subsection{\textbf{Asemblerski jezik}}

Svaki ra\v cunar ima svoju arhitekturu skupa instrukcija (\textbf{Instruction Sat Architecture, ISA}) i ona obuhvata registre, 
instrukcije i druge mogu\' cnosti vidljive programerima na niskom nivou. Ova arhitektura se naziva ma\v sinski jezik. 
Program na ovom nivou apstrakcije je lista binarnih brojeva, po jedan za svaku instrukciju, koji saop\v stavaju koju 
instrukciju treba izvr\v siti i sa kojim operandima. Medjutim, veoma je te\v sko programirati sa binarnim brojevima,
pa svaki ra\v cunar ima i svoj asemblerski jezik, simboli\v cku predstavu arhitekture skupa instrukcija.
Asemblerski jezik ili jednostavno asembler je ni\v zi simboli\v cki jezik, prilagodjen radu ra\v cunara.
Binarni brojevi su sada zamenjeni mnemoni\v ckim oznakamama, tako da je ovaj programski jezik mnogo razumljiviji. 
Osim kratkih re\v ci (ADD, SUB, MUL) koje se koriste za ma\v sinske instrukcije, asembleri dozvoljavaju kori\v s\' cenje 
simboli\v ckih imena za konstante i oznaka koje se odnose na instrukcije i memorijske adrese. Kada se program napisan na 
asemblerskom jeziku prosledi programu zvanom asembler, on ga pretvara u binarni program koji je pogodan za direktno 
izvr\v savanje, na stvarnom hardveru \cite{Arhitektura}. \\ 
\textbf {Za\v sto asembler?} \\
\emph {Asembler je program programa, alat svih alata i on je ozbiljno oru\v zje u rukama pravog programera \cite{Arhitektura}.} \\ 
Asemblerski jezik je najbli\v za forma komunikacije izmedju \v coveka i ma\v sine. Programi u asembleru se odlikuju
mogu\' cno\v s\' cu slanja direktnih komandi procesoru kao i iskori\v s\' cavanju celog dijapazona ra\v cunarske 
arhitekture. Koriste\' ci asembler, programer mo\v ze da ta\v cno prati protok podataka i izvr\v savanje programa.
Programeri \v cesto koriste asembler za kriti\v cne delove programa, ali i  za prepravljanje koda za koji nemamo
izvorni kod. Asemblerski kod \' ce naj\v ce\v s\' ce da se izvr\v sava mnogo br\v ze nego programi napisani u ostalim jezicima.
Proizvodja\v ci hardvera, kao \v sto su Intel i AMD dodaju nove karakteristike svojim procesorima i ve\' cinom je jedini 
na\v cin da im se pristupi preko asemblera \cite{Introduction}. \\

\subsection{\textbf{Arhitektura x86\_64}}

Tokom godina, programeri su koristili x86 asembler da bi pisali kod kriti\v cnih performansi. 64-bitni ra\v cunari su 
zamenili 32-bitne pa se i sam asemblerski kod promenio. x64 je ime za 64-bitne nadogradnje, \textbf{Intelove} i \textbf{AMD}-ove, 32-bitne arhitekture
skupa instrukcija. AMD je predstavio prvu verziju x64 koja se zvala x86\_64, a zatim su je preimenovali u AMD64. 
Intel je svoju implementaciju nazvao IA-32e, a zatim EMT64. Postoje neznatne razlike izmedju dve verzije ali ve\' cina koda 
radi na obe verzije. Treba razlikovati ovu arhitekturu od arhitekture IA-64 (64 bita Intel Itanium). x86\_64 je takodje
poznata kao x64, x86-64 i AMD64 \cite{Wikipedia}.\\
x86\_64 podrzava vece koli\v cine virtuelne i fizi\v cke memorije, dozvoljavaju\' ci programima da \v cuvaju ve\' ce
koli\v cine podataka u memoriji. x86\_64 takodje ima 64-bitne registre op\v ste namene i mnoga druga pobolj\v sanja.
U potpunosti je unazad kompatibilna sa 16-bitnom i 32-bitnom arhitekturom. Zadr\v zava se set instrukcija 16-bitne i 
32-bitne arhitekture, ali uz dodatke. x86 arhitektura je pro\v sirila registre u 64-bitne na veoma sli\v can na\v cin 
na koji su 16-bitni registri pro\v sireni u 32-bitne. Adresiranje se ne menja zna\v cajno u odnosu na 32-bitnu arhitekturu,
samo se pro\v siruje na 64 bita. Arhitektura x86\_64 je kori\v s\' cena pri implementaciji procesora Pentium 4,
Celeron D, Xeon, Pentium Dual-Core, Pentium Extreme Edition, Core 2, Core i3,i5, i7 \cite{Wikipedia}...

\subsection{\textbf{Organizacija registara i na\v cini adresiranja}}

U x86 asembleru postoji 9 primarnih registara: EAX, ECX, EDX, EBX, ESP, EIP, EBP, ESI i EDI. Svi oni mogu da \v cuvaju 32-bitne 
vrednosti. Prva \v cetiri su registri op\v ste namene i koriste se (redom) kao akumulator, broja\v c, registar podataka i 
bazni registar. Njihova glavna uloga je privremeno skladistenje podataka i adresa tokom izvr\v savanja programa.
Drugih 5 registra su takodje registri op\v ste namene ali se ve\' cinom koriste kao pokaziva\v cki i indeksni registri.
Prva dva su od posebnog zna\v caja pri izvr\v savanju programa. Pokaziva\v c steka (ESP) pokazuje na poslednji argument 
koji je stavljen na stek. Va\v zno je znati da stek raste ka ni\v zim adresama. Drugi veoma va\v zan registar je
pokaziva\v c instrukcija (EIP). On pokazuje na narednu instrukciju koju \' ce procesor da izvr\v si. Kako ovo povezujemo sa 
x64 asemblerom? Svi ovi registri su, dakle, i dalje prisutni. Oni su, medjutim, pro\v sireni sa maksimalna 32 bita na 64.
Kao rezultat, oni su sada poznati kao \textbf{RAX, RCX, RDX, RBX, RSP, RBP, RSI, RDI i RIP} registri. Drugim re\v cima, 
E je zamenjeno sa R. Mada to nije u potpunosti ta\v cno. E u imenima registara je i dalje validno u x64 asembleru.
Ono nam omogu\' cuje pristup pristup ni\v zem 32-bitnom delu registra. Isto tako, za RAX, RCX, RDX i RBX registre vazi da
se uklanjanjem slova r moze pristupiti najnizim 16 bitova (tako da RAX bude AX npr.) ili nizim od tih 16 preko AL registra, 
odnosno vi\v sim preko AH (dakle isto kao kod 32-bitnih registara). Kao dodatak, ovim 9 registrima, x86\_64 arhitektura dodaje
registre op\v ste namene \textbf{R8 do R15}. \textbf{RFLAGS} registar skladi\v sti flegove; sadr\v zaj mu se menja u zavisnosti od rezultata 
aritmeti\v ckih operacija. Formiran je dodavanjem vi\v sa 32 bita na registar EFLAGS. Takodje je va\v zno napomenuti 
postojanje registara koji se koriste za operacije u pokretnom zarezu, (eng. floating-point registri). \\
Za skoro svaku instrukciju neophodni su podaci bilo iz memorije, bilo iz registara.
Ve\' cina instrukcije sadr\v zi dva operanda, odredi\v sni i izvori\v sni.\\
Na primer: \textbf{MOV RAX,RBX} \\
Prvi operand je odredi\v sni, a drugi izvori\v sni. Konstanta mo\v ze da bude izvori\v ste, tj. leva vrednost 
ali ne i odredi\v ste. Odredjene instrukcije zahtevaju samo jedan operand: postupno uve\' cavanje, negiranje, pomeranje itd. 
Na primer u instrukciji MUL samo registar RAX mo\v ze da bude odredi\v ste \cite{Introduction}: \\
Na primer: \textbf {MUL RBX} \\

Procesor sme\v sta brojeve po sistemu \textbf{little endian}, sto zna\v ci da se manje zna\v cajan deo re\v ci sme\v sta na ni\v zu 
adresu. Na\v cini adresiranja su zapravo na\v cini na koje instrukcija mo\v ze da pristupi registrima ili memoriji. 
Postoji vi\v se na\v cina adresiranja. Oni koji se naj\v ce\v s\' ce koriste su \cite{Arhitektura}:
\begin{enumerate}
\item {Neposredno}
\item {Direktno}
\item {Indirektno registarsko}
\item {Registarko s pomerajem}
\item {Registarko indeksno}
\item {Registarsko indeksno s pomerajem}
\item {Podrazumevano}
\end{enumerate}

\textbf{Neposredno adresiranje} je na\v cin adresiranja u kome je operand u instrukciji konstantan bajt ili re\v c. Na primer:\\
\textbf {CMP RAX, 50} (poredi vrednost registra RAX sa konstantom 50) \\ \\
\textbf{Direktno adresiranje} podrazumeva da je operand u instrukciji adresa podatka. Na primer:\\
\textbf{ADD RAX, (20)} (dodaje registru RAX vrednost sa memorijske lokacije 20) \\ \\
\textbf{Indirektno registarsko} zna\v ci da se adresa operanda nalazi u nekom od registara RBX, RSI ili RDI. U sva tri 
slu\v caja operand se nalazi u segmentu za podatke. Kada se konstanta postavi ispred registra, adresa se pronalazi dodavanjem
sadr\v zaja registra konstanti. Ovaj tip se naziva \textbf {registarsko adresiranje sa pomerajem}. Zgodan je za adresiranje 
bitova. Na primer:\\
\textbf{SUB [RDI + 4], RAX}  (bazna adresa se nalazi u RDI, uz pomeraj od 4; od te vrednosti se oduzima sadr\v zaj 
registra RAX)\\ \\
\textbf {Registarsko indeksno} podrazumeva da je u RDI po\v cetna adresa niza, a broja\v c u RCX. Na primer:\\
\textbf{CMP RSI, [RDI + RCX]} \\ (poredi vrednost registra RSI sa vrednosti na adresi na koju pokazuje RDI+RCX)
Poslednja dva tipa se mogu kombinovati u \textbf {registarsko indeksno s pomerajem}. Na primer: \\
\textbf {SUB [RDI + 4 * RCX + 4], RAX} (pomeraj je 4 bajta, bazna adresa se nalazi u RDI, indeks u RCX, a faktor skaliranja
je 4; vrednost na ovoj lokaciji se umanjuje za vrednost registra RAX). \\ \\
Na kraju, neke instrukcije koriste \textbf {podrazumevano adresiranje} gde se operand ili operandi podrazumevaju. Na primer: \\
\textbf{PUSH RAX} (stavlja sadr\v zaj registra RAX na stek tako \v sto smanjuje vrednost registra RSP, a zatim kopira sadr\v zaj
registra RAX na lokaciju na koju sada ukazuje RSP. U instrukciji PUSH upotreba registra RSP se podrazumeva). \\ \\


\section{\textbf{INSTRUKCIJE OP\v STE NAMENE}}

%ARITMETICKE INSTRUKCIJE
\subsection{\textbf{Aritmeti\v cke instrukcije}} 
Aritmeti\v cke instrukcije izvr\v savaju aritmeti\v cke operacije i zahtevaju razli\v cit broj operanada,
u zavisnosti od same operacije.
Prvi operand je odredi\v sni i mora biti registar ili memorijska lokacija. Drugi operand mo\v ze biti ili
memorijska lokacija, registar ili konstantna vrednost. Barem jedan od ova mora biti registar zato \v sto 
operacije ne smeju da koriste memorijske lokacije kao oba operanda \cite{x86Assembly}. \\ \\
\begin{enumerate}
\item{Instrukcija \textbf{ADD} vr\v si \textbf{sabiranje} i ima dva operanda:} \\
\textbf{ADD RAX, 4} (sabiramo vrednost registra RAX sa konstantom 4 i rezultat \' ce biti sme\v sten u registar RAX) \\
\item{Instrukcija \textbf{SUB} vr\v si \textbf{oduzimanje} i ima dva operanda:} \\
\textbf{SUB RBX, X}  (oduzimamo vrednost sa memorijske lokacije X od vrednosti registra RBX i rezultat sme\v stamo u RBX) \\ \
\item{Instrukcija \textbf{ADC} vr\v si \textbf{sabiranje s prenosom} i ima dva operanda: }
\textbf{ADC RAX, 4}  (sabiranje sa prenosom, vrednost eng. carry flag-a se dodaje na rezultat sabiranja) \\ 
\item{Instrukcija \textbf{SBB} vr\v si \textbf{oduzimanje s pozajmicom} i ima dva operanda:}
\textbf{SBB RBX, X}  (oduzimanje sa pozajmicom, vrednost eng. carry flag-a se oduzima od razlike operanada) \\ 
\item {Razlikujemo dve instrukcije za mnozenje: \textbf{MUL} za neozna\v cene i \textbf{IMUL} za ozna\v cene brojeve. Instrukcija MUL ima samo jedan operand i taj operand se mno\v zi sa 
sadr\v zajem registra  AL, AX, EAX ili RAX (u zavisnosti od \v sirine operanda). Rezultat se sme\v sta u registre AX, DX:AX,
EDX:EAX, RDX:RAX (dvota\v cka ozna\v cava da se prvi registar odnosi na vi\v se bitove podataka). IMUL mo\v ze imati jedan 
operand i u tom slu\v caju je analogna instrukciji MUL, ali mo\v ze imati i dva i tri operanda. Na primer:} \\
\textbf{MUL RBX} (sadr\v zaj registra RAX se mno\v zi sa sadr\v zajem registra RBX i rezultat se sme\v sta u RDX:RAX)\\ 
\item {\v Sto se deljenja ti\v ce, ovde takodje imamo dve instrukcije: \textbf{DIV} se odnosi na deljenje neozna\v cenih, 
a \textbf{IDIV} na deljenje ozna\v cenih brojeva. Instrukcija DIV ima jedan operand. Sadr\v zaj 
registra AX, DX:AX, EDX:EAX, RDX:RAX se deli vredno\v scu operanda koja je 8-bitna, 16-bitna, 32-bitna i 64-bitna i 
koli\v cnik se sme\v sta (redom) u: AL, AX, EAX i RAX, a ostatak pri deljenju u AH, DX, EDX I RDX. IDIV je analogna 
instrukciji DIV i nema ekstenzije kao IMUL. Na primer:} \\
\textbf{DIV RCX} (sadr\v zaj registra RDX:RAX se deli sadr\v zajem registra RCX i koli\v cnik se sme\v sta u RAX 
a ostatak u RDX) \\ 
\item {Instrukcija \textbf{NEG}} vr\v si \textbf{negaciju} i ima jedan operand kome menja znak. Na primer: \\
\textbf{NEG X} (vrednosti na memorijskoj lokaciji x je promenjen znak)\\ 
\item {Instrukcije \textbf{INC} i \textbf{DEC} vr\v se (redom) \textbf{uvecavanje i smanjivanje argumenta za jedan} i imaju po jedan operand.
Inkrementacija radi mnogo br\v ze nego \textbf {ADD ARG, 1}. Isto tako, dekrementacija radi mnogo br\v ze nego 
\textbf{SUB ARG, 1}. Na primer:} \\
\textbf{INC RDX} (uve\' cava vrednost registra RDX za 1)\\
\textbf{DEC RDX} (smanjuje vrednost registra RDX za 1)\\ 
\end{enumerate}

%LOGICKE INSTRUKICJE
\subsection{\textbf{Logi\v cke instrukcije}}
Logi\v cke instrukcije izvode logi\v cke operacije nad operandima. Sve osim bitske negacije prihvataju dva operanda, 
dok negacija prihvata jedan. Razlikujemo slede\' ce bitovske logi\v cke operacije \cite{x86Assembly}:
\begin{enumerate}
\item {\textbf{Bitovska konjukcija (AND): AND RDX, RCX}}
\item {\textbf{Bitovska disjunkcija (OR): OR RDX, RCX}}
\item {\textbf{Bitovska ekskluzivna disjunkcija (XOR): XOR RDX, RCX}}
\item {\textbf{Bitovkska negacija (NOT): NOT RDX}}
\end{enumerate}


%INSTRUKCIJE ZA SIFTOVANJE I ROTACIJU
\subsection{\textbf{Instrukcije za \v siftovanje i rotacije}}
Sve instrukcije za \v siftovanje i rotacije prihvataju po jedan operand \cite{x86Assembly}. \\ 
\begin{enumerate}
\item{\textbf{Logi\v cke instrukcije za \v siftovanje}}\\
Logi\v cko \v siftovanje se koristi za neozna\v cene brojeve.
U slu\v caju logickog pomeranja udesno, krajnje desni bit se zanemaruje, a po\v cetni bit rezultata 
se uvek popunjava nulom, a u slu\v caju logi\v ckog pomeranja ulevo, po\v cetni bit argumenta se zanemaruje, a na 
zavr\v sna mesta se upisuje nula. 

\begin{enumerate}
\item{\textbf{\v Siftovanje udesno (SHR): SHR RDX}}
\item{\textbf{\v Siftovanje ulevo (SHL): SHL RDX}}
\end{enumerate}

\item{\textbf{Aritmeti\v cke instrukcije za \v siftovanje}}\\
Artimeti\v cko pomeranje se koristi za ozna\v cene brojeve u potpunom komplementu.
U slu\v caju aritmeti\v ckog pomeranja udesno, krajnje desni bit se zanemaruje, a po\v cetni bit se popunjava
vode\' cim bitom argumenta koji predstavlja znak. A \v sto se ti\v ce pomeranja ulevo, po\v cetni bit argumenta se 
zanemaruje, dok se na zavr\v sno mesto upisuju nule.

\begin{enumerate}
\item{\textbf{\v Siftovanje u desno (SAR): SAR RAX}}
\item{\textbf{\v Siftovanje u levo (SAL): SAL RAX}}
\end{enumerate}

\item{\textbf{Instrukcije za \v siftovanje sa prenosom}} \\
Primenjujemo uobi\v cajeno logi\v cko \v siftovanje, s tim \v sto se bitovi koji se ina\v ce zanemaruju, \v salju u eng. carry 
flag. Primer:

\begin {enumerate}
\item {\textbf{\v Siftovanje u desno sa prenosom (SCR): SCR RAX}}
\item {\textbf{\v Siftovanje u levo sa prenosom (SCL): SCL RAX}}
\end{enumerate}

\item{\textbf{Instrukcije za rotaciju}} \\
Rotacija udesno je \v siftovanje udesno, s tim \v sto bit koji gubimo sa po\v cetka vra\' camo na kraj, dok je rotacija
ulevo siftovanje ulevo, sa vra\' canjem izgubljenog na kraj.
\begin{enumerate}
\item {\textbf{rotacija udesno (ROR): ROR RAX}}
\item {\textbf{rotacija ulevo (ROL): ROL RAX}}
\end {enumerate}
\end{enumerate}

%INSTRUKCIJE ZA TRANSFER PODATAKA
\subsection{\textbf{Instrukcije za transfer podataka}}
Kopiranje podataka sa jednog mesta na drugo je najosnovnije od svih operacija. Pod kopiranjem podrazumevamo stvaranje novog objekta koji ima 
identi\v can raspored bitova kao original. Instrukcije za preme\v stanje podataka bi bilo bolje zvati instrukcije za dupliranje podataka, ali
se izraz preme\v stanje podataka ve\' c previ\v se odoma\' cio. Postoje dva razloga zbog kojih se podaci kopiraju s jednog mesta na drugo. Jedan je 
dodeljivanje vrednosti promenljivama, a drugi razlog je efikasnije pristupanje i kori\v s\' cenje (brojne instrukcije mogu da pristupe
promenljivama samo ako se one nalaze u registrima) \cite{x86Assembly}. \\ 
\begin{enumerate}
\item{Instrukcija \textbf {MOV} kopira vrednost drugog operanda u prvi, pri \v cemu se nijedan fleg ne menja. Na primer:} \\
\textbf{MOV RAX, 6} (kopiramo konstantu 6 u registar RAX)
\item{Instrukcija \textbf {XCHG} za zamenu podataka zamenjuje vrednosti prvog i drugog operanda, pri \v cemu se takodje nijedan fleg ne menja. Na primer:} \\
\textbf{XCHG RAX, RBX} (zamenjujemo vrednosti u registrima RAX i RBX) 
\item{Takodje postoje instrukcije koje pri kopiranju drugog operanda u prvi, nepopunjena mesta u prvom argumentu popunjavaju nulama (instrukcija
je korisna za kopiranje malih neozna\v cenih vrednosti u ve\' ci registar) ili znakom drugog argumenta (instrukcija je korisna za kopiranje malih 
ozna\v cenih brojeva)}. Na primer:\\ 
\textbf{MOVZ RAX, 5} (prvi slu\v caj)\\
\textbf{MOVS RAX, 5} (drugi slu\v caj)
\item{Instrukcija \textbf{MOVSB} nema nijedan operand i kopira jedan bajt sa lokacije koja se nalazi u RSI registru na lokaciju koja se nalazi u RDI. Instrukcija
takodje ne menja flegove.}
\item{\textbf{MOVSW} je instrukcija koja nema nijedan operand i kopira dva bajta (jednu re\v c) iz RSI u RDI.}
\end{enumerate}

%INSTRUKCIJE ZA KONTROLU TOKA
\subsection{\textbf{Instrukcije za kontrolu toka}}
U skoro svim programskim jezicima mo\v ze da se pri izvr\v savanju instrukcija menja njihov redosled, pa ni asembler nije izuzetak.
Pokaziva\v c instrukcija (RIP) registar sadr\v zi adresu naredne instrukcije koja \' ce da se izvr\v si. Da bi se tok kontrole promenio, 
programer mora da modifikuje vrednost registra RIP. Upravo tu, glavnu ulogu igraju instrukcije za kontrolu toka \cite{x86Assembly}.

\begin{enumerate} %glavni begin

\item{\textbf{Instrukcije za poredjenje}} %1
	\begin{enumerate} 
	\item{\textbf{TEST} je instrukcija koja izvr\v sava bitovsku konjukciju nad dva operanda i postavlja flegove, ali ne \v cuva rezultat Na primer:} \\
	\textbf{TEST RAX, RBX} 
	\item{\textbf{CMP} je instrukcija koja vr\v si oduzimanje izmedju dva operanda, postavlja flegove, ali ne \v cuva rezultat. Na primer:} \\
	\textbf{CMP RAX, RBX}
	\end {enumerate}


\item{\textbf{Skokovi}} %2
\begin{enumerate}
\item{\textbf{Bezuslovni skok}}\\ 
    Instrukcija \textbf{JMP} postavlja u registar RIP adresu naredne instrukcije koja se izvr\v sava. Na primer: \\
	\textbf{JMP LOC} (ska\v ce na izvr\v savanje instrukcije koja se nalazi na adresi LOC)\\
\item{\textbf{Skok ako je jednako}}\\
    Instrukcija \textbf{JE} postavlja u RIP adresu naredne instrukcije, ako su operandi prethodne CMP instrukcije jednaki.\\ 
	\textbf{MOV RCX, 5}\\
	\textbf{MOV RDX, 5}\\
	\textbf{CMP RCX, RDX}\\
	\textbf{JE EQUAL}\\
	\# ako se ne sko\v ci na labelu EQUAL, onda to zna\v ci da  5 i 5 nisu jednaki.\\
	\textbf{EQUAL}:\\
	\# ako se sko\v cilo ovde, onda to zna\v ci da su 5 i 5 jednaki. \\

\item{\textbf{Skok ako nije jednako}}\\
	Instrukcija \textbf{JNE} stavlja u RIP registar adresu naredne instrukcije, ako operandi prethodne CMP instrukcije nisu jednaki. \\

\item{\textbf{Skok ako je ve\' ce}}\\
	\begin{enumerate}
	 \item{Instrukcija \textbf{JG} u registar RIP stavlja adresu naredne instrukcije, ako je prvi operand prethodne CMP instrukcije ve\' ci od drugog.
	(izvodi se poredjenje ozna\v cenih brojeva).}\\
	
	\item {Instrukcija \textbf{JGE} stavlja u RIP adresu naredne instrukcije, ako je prvi operand prethodne CMP instrukcije ve\' ci ili jednak drugom
	(izvodi se poredjenje ozna\v cenih brojeva).}\\

	\item {Instrukcija \textbf{JA} stavlja u RIP adresu naredne instrukcije, ako je prvi operand prethodne CMP instrukcije ve\' ci od drugog.
	(izvodi se poredjenje neozna\v cenih brojeva).}\\

	\item {Instrukcija \textbf{JAE} stavlja u RIP adresu naredne instrukcije, ako je prvi operand prethodne CMP instrukcije ve\' ci ili jednak drugom
	(izvodi se poredjenje neozna\v cenih brojeva).}\\
	\end{enumerate}%skok ako je vece

\item{\textbf{Skok ako je manje}}
	\begin{enumerate}
	\item{Instrukcija \textbf{JL} u registar RIP stavlja adresu naredne instrukcije, ako je prvi operand prethodne CMP instrukcije manji od drugog.
	(izvodi se poredjenje ozna\v cenih brojeva).}\\
	
	\item{Instrukcija \textbf{JLE} stavlja u RIP adresu naredne instrukcije, ako je prvi operand prethodne CMP instrukcije manji ili jednak drugom
	(izvodi se poredjenje ozna\v cenih brojeva).}\\

	\item{Instrukcija \textbf{JB} stavlja u RIP adresu naredne instrukcije, ako je prvi operand prethodne CMP instrukcije manji od drugog.
	(izvodi se poredjenje neozna\v cenih brojeva).}\\

	\item{Instrukcija \textbf{JBE} stavlja u RIP adresu naredne instrukcije, ako je prvi operand prethodne CMP instrukcije manji ili jednak drugom
	(izvodi se poredjenje neozna\v cenih brojeva).}\\
	\end{enumerate}%skok ako je manje

\item{\textbf{Skok ako postoji prekora\v cenje}} \\
	Instrukcija \textbf{JO} u registar RIP stavlja adresu naredne instrukcije, ako je postavljen bit prekora\v cenja pri prethodnoj aritmeti\v ckoj operaciji.\\

\item{\textbf{Skok u slu\v caju nule}}
	\begin{enumerate}
	
	\item{Instrukcija \textbf{JNZ} u registar RIP stavlja adresu naredne instrukcije, ako nije postavljen eng. zero flag pri prethodnoj aritmeti\v ckoj operaciji.
	JNZ instrukcija je identi\v cna instrukciji JNE.}\\

	\item{Instrukcija \textbf{JZ} stavlja u RIP adresu naredne instrukcije,ako je postavljen eng. zero flag pri prethodnoj aritmeti\v ckoj operaciji.
	JZ je identi\v cna JE.}\\
	\end{enumerate}%skok u slucaju nule

\end{enumerate} %kraj za skokove



\item{\textbf{Instrukcije za poziv procedura}} \\
Instrukcija \textbf{CALL} bezuslovno ska\v ce na adresu koja je zadata kao operand, ali prethodno stavlja na vrh steka
adresu povratka iz RIP registra.\\
Instrukcija \textbf{RET} zavr\v sava potrogram, ona skida sa steka povratnu adresu, stavlja je u RIP registar 
i vr\v si bezuslovan skok na tu adresu.
Vrlo je va\v zno voditi ra\v cuna da ako se na stek postavljaju neki drugi podaci, da se oni pre poziva RET instrukcije
skinu sa steka, kao i da se gre\v skom ne izmeni sadr\v zaj povratne adrese na steku.


%INSTRUKCIJE BROJACKE PETLJE
\item{\textbf{Instrukcije broja\v cke petlje}}\\
Sve instrukcije primaju jedan argument koji je adresa memorijske lokacije, umanjuju vrednost 
registra RCX i ska\v cu na zadatu adresu u zavisnosti od uslova.
\begin{enumerate}

\item{Instrukcija \textbf{LOOP}} \\
	Ima samo jedan argument koji predstavlja adresu memorijske lokacije.
	Ona umanjuje vrednost registra RCX za jedan i ska\v ce na zadatu adresu sve dok vrednost registra RAX ne bude nula.
	LOOP ne postavlja flegove. Na primer: \\ \\
	
	\textbf{MOV RCX, 5} \\
	\textbf{START\_LOOP:} \\
	\#kod bi se ovde izvr\v savao 5 puta\\
	\textbf{LOOP START\_LOOP:} \\ \\

\item{Instrukcija \textbf{LOOPX}}\\
	Ove LOOP instrukcije umanjuju vrednost registra RCX i ska\v ce na adresu zadatu operandom
	ako je njihov uslov zadovoljen, tj. ako je postavljen odgovaraju\' ci fleg.
	\begin{enumerate}
	\item{\textbf{LOOPE}} ska\v ce ako je jednako
	\item{\textbf{LOOPZ}} ska\v ce ako je nula
	\item{\textbf{LOOPNE}} ska\v ce ako nije jednako
	\item{\textbf{LOOPNZ}} ska\v ce ako nije nula
	\end{enumerate}
\end{enumerate} %instrukcije brojacke petlje


\item{Instrukcije \textbf{ENTER} i \textbf{LEAVE}} \\
Instrukcija \textbf{ENTER} pravi stek okvir sa odredjenom koli\v cinom 
prostora alociranom na steku. Obi\v cno se taj prostor rezervi\v se za lokalne promenljive 
koje se koriste u procedurama. Na primer: \\
\textbf{ENTER 0,0} (u ovom slu\v caju se ne rezervise prostor za lokalne promenljive na steku)\\
Instrukcija \textbf{LEAVE} uni\v stava trenutni stek okvir i restaurira prethodni okvir. Na primer: \\
\textbf{MOV RSP, RBP}\\
\textbf{POP RBP}\\ \\
Sadr\v zaj RSP registra je adresa vrha steka, ali se on tokom rada funkcije mo\v ze promeniti, sto nam u tom 
slu\v caju ote\v zava adresiranje argumenata funkcije. Standardni na\v cin pristupa argumentima funkcije je preko registra RBP.
Po ulasku u funkciju, ovaj registar se stavlja na stek (da bi se njegova vrednost sa\v cuvala) i u njega se kopira vrednost
registra RSP, tj. adresa vrha steka \' ce nam se sad nalaziti i u RBP registru. Sadr\v zaj RBP registra se posle ne menja 
u toku rada funkcije i preko njega mi pristupamo njenim argumentima. Na vrhu steka je sadr\v zaj RBP registra, ispod njega 
je adresa povratka iz funkcije, a onda redom argumenti funkcije kojima se pristupa tako \v sto na adresu RBP dodamo 
odredjen broj bajtova u zavisnosti gde se na steku ti argumenti nalaze (jer stek raste ka ni\v zim adresama). Npr. ako 
ho\' cemo da pristupimo prvom argumentu funkcije dodajemo 8 bajtova na RBP, za drugi argument dodajemo 12 bajtova itd.
Na kraju funkcije mora se skinuti sadr\v zaj registra RBP sa steka i tek onda mo\v ze da se pozove RET instrukcija.
Gore navedenim naredbama postavili smo sadr\v zaj RBP registra u registar RSP (u kom se \v cuva adresa vrha steka) i 
skinuli registar RBP sa steka kako bi RET instrukcija skinula povratnu adresu i kako bi program mogao da
nastavi da se izvr\v sava od te adrese.

%DRUGE INSTRUKCIJE KONTROLE
\item{\textbf{Druge instrukcije kontrole}}
\begin{enumerate}
\item{\textbf{HLT} instrukcija zaustavlja procesor} 
\item{\textbf{NOP} zna\v ci da nema operacije. Ova instrukcija ne radi nista, tj. samo tro\v si instrukcioni ciklus.\v Cesto je zamenjuje instrukcija: \textbf{XCHG RAX, RAX.}} 
\item{\textbf{LOCK} instrukcija isti\v ce procesorov signal LOCK koji je istaknut tokom operacije koja ide uz instrukciju.}
\item{\textbf{WAIT} instrukcija \v ceka da procesor zavr\v si svoju poslednju operaciju.} \\
\end{enumerate}


\end{enumerate} %instrukcija kontrole toka



%INSTRUKCIJE ZA RAD SA STEKOM
\subsection{\textbf{Instrukcije za rad sa stekom}}
Instrukcije primaju po jedan argument, osim instrukcija PUSHA i POPA koje nemaju argumente \cite{x86Assembly}.
\begin{enumerate}
\item{\textbf{PUSH}}\\
	Ova instrukcija smanjuje za 1 pokaziva\v c na stek i podatak koji se nalazi u argumentu stavlja na lokaciju na koju pokazuje pokaziva\v c na stek.
\item{\textbf{POP}}\\
	Ova instukcija stavlja podatke koji se nalaze na lokaciji na koju pokazuje pokaziva\v c na stek u argument, a zatim povecava pokaziva\v c na stek za 1.\\ \\
	\textbf{MOV RAX, 5}\\
	\textbf{MOV RBX, 6}\\
	\textbf{PUSH RAX} (posle ove instrukcije, stek bi bio: [5])\\    
	\textbf{PUSH RBX} (posle ove instrukcije, stek bi bio: [6] [5])\\
	\textbf{POP RAX}  (ova instrukcija vrednost koja je poslednja stavljena na stek[6], kopira u RAX. Posle ove operacije, stek bi bio: [5])\\
	\textbf{POP RBX} (vrednost registra RBX bi bila 5, a stek bi bio prazan)\\
\item{\textbf{PUSHF}}\\
	Ova instrukcija smanjuje pokaziva\v c na stek za 1, a zatim na lokaciju na koju pokazuje pokaziva\v c na stek, stavlja sadr\v zaj fleg registra.
\item{\textbf{POPF}}\\
	Ova instrukcija u fleg regitar stavlja sadrzaj memorijske adrese na koju pokazuje pokaziva\v c na stek, a zatim pokaziva\v c na stek uve\' cava za 1.
\item {Instrukcijama \textbf{PUSHA} i \textbf{POPA} se mogu brzo sa\v cuvati svi registri op\v ste namene}

\end{enumerate}


%INSTRUKCIJE ZA RAD SA FLEGOVIMA 
\subsection{\textbf{Instrukcije za rad sa fleg-ovima}}
Dok fleg registar slu\v zi da prijavi rezultat izvr\v senih instrukcija (prekora\v cenje, prenos itd.),
takodje sadr\v zi i flegove koji uti\v cu na operacije procesora. Ovi flegovi bivaju postavljeni i poni\v steni 
posebnim instrukcijama \cite{x86Assembly}.
\begin{enumerate}

\item{\textbf{Fleg za prekide}} govori procesoru treba li ili ne da prihvati hardverski prekid.
	\begin{enumerate}
	\item{\textbf{STI}}\\
	Ova instrukcija postavlja fleg za prekide; u tom slu\v caju procesor prihvata zahtev za prekid.
	\item{\textbf{CLI}}\\ 
	Ova instrukcija poni\v stava fleg za prekide.
	\end{enumerate}
	
\item{\textbf{Fleg za prenos}} se obi\v cno postavlja nakon aritmeti\v ckih instrukcija, ali mo\v ze biti postavljen ili poni\v sten i ru\v cno.
	\begin{enumerate}
	\item{\textbf{STC}}\\
	Ova instrukcija postavlja fleg za prenos.
	\item{\textbf{CLC}}\\ 
	Ova instrukcija poni\v stava fleg za prenos.
	\item{\textbf{CMC}}\\
	Ova instrukcija komplementira fleg za prenos.
	\end{enumerate}
	
\item{\textbf{Ostali flag-ovi}}
	\begin{enumerate}
	\item{\textbf{SAHF}}\\
	Ova instrukcija \v cuva sadr\v zaj AH registra u ni\v zim bajtovima fleg registra.
	\item{\textbf{LAHF}}\\ 
	Ova instrukcija stavlja u AH registar ono \v sto se nalazi u ni\v zim bajtovima fleg registra.
	\end{enumerate}

\end{enumerate} %inst za rad sa flagovima


%ULAZNO-IZLAZNE INSTRUKCIJE
\subsection{\textbf{Ulazno-izlazne instrukcije}}
Instrukcije primaju po dva operanda \cite{x86Assembly}.
\begin{enumerate}
\item{\textbf{IN}}\\
Instrukcija IN skoro uvek radi sa operandima AX i DX (tj. EAX, EDX ili RAX, RDX). DX \v cesto drzi adresu porta za \v citanje, a AX prima podatke od porta.
U za\v sti\' cenom re\v zimu operativnih sistema, IN instrukcija je zaklju\v cana, pa je obi\v cni korisnici ne mogu koristiti.\\
\item{\textbf{OUT}}\\
Instrukcija OUT je veoma sli\v cna instrukciji IN. OUT stavlja podatke iz registra koje je u prvom operandu, na port dat u drugom operandu. U za\v sti\' cenom
re\v zimu, instrukcija OUT je zaklju\v cana, pa ni nju obi\v cni korisnici ne mogu koristiti.\\
\end{enumerate} %Ulazno\izlazne instrukcije


%SISTEMSKE INSTRUKCIJE
\subsection{\textbf{Sistemske instrukcije}}
\begin{enumerate}
\item{\textbf{SYSENTER}}\\
Ova instrukcija uvodi procesor u za\v sti\' ceni re\v zim.
\item{\textbf{SYSEXIT}} \\
Ova instrukcija izvla\v ci procesor iz za\v sti\' cenog re\v zima i uvodi ga u re\v zim za korisnika.
\end{enumerate} %Sistemske instrukcije

\subsection{\textbf{Instrukcije za generisanje prekida}}
Sistem prekida predstavlja jedan od mehanizama za upravljanje tokom rada procesora. U osnovne
uloge sistema prekida spadaju komunikacija sa U-I uredjajima i upotreba osnovnih usluga operativnog sistema.
Prekidi se mogu izazvati hardverski ili softverski. Kod procesora x86 \textbf{INT}
je instrukcija koja poziva odgovaraju\' cu prekidnu proceduru. Ima jedan operand - broj prekida. 
Kada se opslu\v zivanje prekida dovr\v si, program nastavlja sa izvr\v savanjem od mesta na kom je 
prekinut \cite{x86Assembly}. Na primer:\\
\textbf{INT 0x0A} (poziva prekid broj 10) 


\section{\textbf{ZAKLJU\v CAK}}

Pri samom u\v cenju asemblera, ono od \v cega kre\' cemo jeste zapravo u\v cenje o skupu instrukcija. 
Ve\' cina x86 instrukcija je validna u 64-bitnoj nadogradnji, s tim da su neke retko kori\v s\' cene 
instrukcije izba\v cene iz upotrebe, a pridodate su nove kao na primer \cite{Wikipedia2}: \\
\begin{enumerate}
\item{\textbf{CDQE}} koja konvertuje eng. dword (EAX) u eng. qword (RAX).
\item{\textbf{CQO}} koja konvertuje eng. qword (RAX) u eng. oword (RDX:RAX).
\item{\textbf{MOVSQ}} koja kopira qword iz RSI registra u RDI.
\item{\textbf{CMPSQ}} koja poredi qword iz RDI sa RAX registrom.
\item{\textbf{SCASQ}} koja u\v citava qword u RDI, a zatim uporedjuje qword iz RDI sa RAX registrom.
\item{\textbf{LODSQ}} koja popunjava RAX registar sa qword-om iz RSI registra.
\item{\textbf{STOSQ}} koja skladi\v sti qword u RDI iz RAX registra.
\item{\textbf{MOVSXD}} koja pro\v siruje eng. dword u qword tako \v sto preostala mesta popunjava znakom dword-a.
\item{\textbf{JRCXZ}} koja ska\v ce ako je RCX nula.
\item{\textbf{POPFQ}} koja skida sa steka RFLAGS registar.
\item{\textbf{PUSHFQ}} koja stavlja na stek RFLAGS registar.
\item{\textbf{IRETQ	}} 64-bitni povratak iz prekida.	
\end{enumerate}

Srce svakog ra\v cunara je skup instrukcija koje mo\v ze da izvr\v sava. 
Da bismo stvarno razumeli ra\v cunar, neophodno je da prvo dobro razumemo njegov skup instrukcija.
Nije retko da ih ra\v cunar u svom repertoaru ima i preko 200. Ovde su prikazane one koje se naj\v ce\v s\' ce koriste 
i predstavljaju dobru osnovu za dalje i detaljnije bavljenje istim.

\newpage
\addcontentsline{toc}{section}{Literatura}
\appendix
\bibliography{seminarski} 
\bibliographystyle{plain}



\end{document}




























































